\documentclass[14pt]{extarticle}
\usepackage{graphicx}
\usepackage{amsmath,amssymb,mathtools}
\usepackage[thinc]{esdiff}
\usepackage{bbm}
\usepackage{amsthm}
\usepackage{hyperref}
\newcommand{\bb}{\mathbbm}
\title{\vspace{-3cm}MA-105 Tutorial-8 Solutions}
\author{Daksh Maahor}
\date{September 2025}

\begin{document}
\maketitle

\bigskip

\begin{enumerate}

%%%%%%%%%%%%%%%%%%%%%%%%%%%%%%%%%%%%%%%%%%%%%%%%%%%%%%%%%%%%
\item \textbf{Compute the arc-length of the cycloid}
\[
\vec{r}(t) = \bigl(a(t-\sin t),\; a(1-\cos t)\bigr)\qquad 0\le t\le 2\pi
\]

\textbf{Sol.}
\begin{align*}
\vec{r}(t) &= \bigl(a(t-\sin t),\; a(1-\cos t)\bigr)\\
\Rightarrow\quad \vec{r}'(t) &= \bigl(a(1-\cos t),\; a\sin t\bigr)
\end{align*}

Now:

\begin{align*}
\|\vec{r}'(t)\|^2 
&= \bigl(a(1-\cos t)\bigr)^2 + \bigl(a\sin t\bigr)^2 \\
&= a^2(1-\cos t)^2 + a^2\sin^2 t \\
&= a^2\Bigl((1-\cos t)^2 + \sin^2 t\Bigr)
\end{align*}

\begin{align*}
(1-\cos t)^2 + \sin^2 t
&= (1 - 2\cos t + \cos^2 t) + \sin^2 t \\
&= 1 - 2\cos t + (\cos^2 t + \sin^2 t) \\
&= 1 - 2\cos t + 1 \\
&= 2 - 2\cos t \\
&= 2(1-\cos t)
\end{align*}

Hence
\[
\|\vec{r}'(t)\|^2 = a^2\cdot 2(1-\cos t) = 2a^2(1-\cos t)
\]

Use the half-angle identity $1-\cos t = 2\sin^2\!\dfrac{t}{2}$ to rewrite:

\[
\|\vec{r}'(t)\|^2 = 2a^2 \cdot 2\sin^2\!\frac{t}{2} = 4a^2 \sin^2\!\frac{t}{2}
\]

\[
\|\vec{r}'(t)\| = \sqrt{4a^2 \sin^2\!\frac{t}{2}} = 2a\left|\sin\frac{t}{2}\right|
\]

The arc-length is
\[
L=\int_{0}^{2\pi}\|\vec{r}'(t)\|\,dt
= \int_{0}^{2\pi} 2a\left|\sin\frac{t}{2}\right|\,dt
\]

\[
L = \int_{u=0}^{\pi} 2a \left|\sin u\right|\cdot 2\,du = 4a\int_{0}^{\pi} |\sin u|\,du = 8a
\]

\noindent \textbf{Final answer:} $\displaystyle L = 8a$

\newpage
%%%%%%%%%%%%%%%%%%%%%%%%%%%%%%%%%%%%%%%%%%%%%%%%%%%%%%%%%%%%
\item \textbf{Parametrize the ellipse $\dfrac{x^2}{a^2}+\dfrac{y^2}{b^2}=1$ and set up the perimeter integral in terms of the eccentricity $e=\sqrt{1-\dfrac{b^2}{a^2}}$ assuming $b<a$}

\textbf{Sol.}

A standard parametrization of the ellipse is
\[
x=a\cos\theta\qquad y=b\sin\theta\qquad \theta\in[0,2\pi]
\]

Differentiate component-wise:
\[
\vec{r}(\theta)=(a\cos\theta,\; b\sin\theta)\qquad
\vec{r}'(\theta)=(-a\sin\theta,\; b\cos\theta)
\]
\[
\|\vec{r}'(\theta)\|^2 = a^2\sin^2\theta + b^2\cos^2\theta
\]

Therefore the perimeter (circumference) is
\[
L=\int_{0}^{2\pi}\|\vec{r}'(\theta)\|\,d\theta
= \int_{0}^{2\pi} \sqrt{a^2\sin^2\theta + b^2\cos^2\theta}\, d\theta
\]

\textbf{Symmetry reduction} The integrand has period $\pi$ and symmetry in the four quadrants hence
\[
L = 4\int_{0}^{\pi/2}\sqrt{a^2\sin^2\theta + b^2\cos^2\theta}\, d\theta
\]

Write $b^2 = a^2(1-e^2)$ where $0\le e<1$ since $b<a$ Then

\begin{align*}
a^2\sin^2\theta + b^2\cos^2\theta
&= a^2\sin^2\theta + a^2(1-e^2)\cos^2\theta \\
&= a^2\bigl(\sin^2\theta + (1-e^2)\cos^2\theta\bigr) \\
&= a^2\bigl(\sin^2\theta + \cos^2\theta - e^2\cos^2\theta\bigr) \\
&= a^2\bigl(1 - e^2\cos^2\theta\bigr)
\end{align*}

Therefore the integrand simplifies to
\[
\|\vec{r}'(\theta)\| = a\sqrt{1 - e^2\cos^2\theta}
\]

Thus the perimeter becomes
\[
\boxed{\,L = 4a\int_{0}^{\pi/2} \sqrt{1 - e^2\cos^2\theta}\, d\theta\, }
\]

(One often sees the alternative form with $\sin^2$ inside by the substitution $\theta\mapsto \tfrac{\pi}{2}-\theta$
\[
\boxed{L = 4a\int_{0}^{\pi/2} \sqrt{1 - e^2\sin^2\theta}\, d\theta}
\]
which is the standard complete elliptic integral of the second kind)

\medskip

\noindent \textbf{Remark:} This integral cannot be expressed in elementary functions for arbitrary $e$ it is denoted
\[
L = 4a\,E(e)
\]
where $E(e)=\int_{0}^{\pi/2}\sqrt{1-e^2\sin^2\theta}\,d\theta$ is the complete elliptic integral of the second kind

\newpage
%%%%%%%%%%%%%%%%%%%%%%%%%%%%%%%%%%%%%%%%%%%%%%%%%%%%%%%%%%%%
\item \textbf{If $\hat a$ and $\hat b$ are two unit vectors in $\bb R^2$ such that $\hat a\times\hat b\neq\mathbf{0}$ show that the set}
\[
\{\hat a\cos t + \hat b\sin t \mid t\in[0,2\pi]\}
\]
\textbf{is an ellipse}

\textbf{Sol.}

Write the unit vectors in coordinates:
\[
\hat a = (a_1,a_2)\qquad \hat b=(b_1,b_2)
\]
The condition $\hat a\times\hat b\neq\mathbf{0}$ in $\bb R^2$ means the scalar (2D) cross product (or determinant)
\[
D := a_1 b_2 - a_2 b_1 \neq 0
\]
so $\hat a$ and $\hat b$ are linearly independent

Form the parametric coordinates $(x,y)$ for $t\in[0,2\pi]$:
\[
x(t) = a_1\cos t + b_1\sin t\qquad
y(t) = a_2\cos t + b_2\sin t
\]

We solve these two linear equations for $\cos t$ and $\sin t$ Treat it as a $2\times2$ linear system in the unknowns $\cos t,\sin t$

\[
\begin{pmatrix} x \\ y \end{pmatrix}
=
\begin{pmatrix} a_1 & b_1 \\ a_2 & b_2 \end{pmatrix}
\begin{pmatrix} \cos t \\ \sin t \end{pmatrix}
\]

Since the coefficient matrix has determinant $D\neq 0$ it is invertible Using Cramer's rule the solutions are

\begin{align*}
\cos t &= \frac{1}{D}\det\begin{pmatrix} x & b_1 \\ y & b_2 \end{pmatrix}
= \frac{b_2 x - b_1 y}{D}\\[6pt]
\sin t &= \frac{1}{D}\det\begin{pmatrix} a_1 & x \\ a_2 & y \end{pmatrix}
= \frac{a_1 y - a_2 x}{D}
\end{align*}

Now impose the trigonometric identity $\cos^2 t + \sin^2 t = 1$ Substitute the expressions above:

\[
\left(\frac{b_2 x - b_1 y}{D}\right)^2 + \left(\frac{a_1 y - a_2 x}{D}\right)^2 = 1
\]

Multiply by $D^2$ and expand to obtain a quadratic equation in $(x,y)$:

\[
(b_2 x - b_1 y)^2 + (a_1 y - a_2 x)^2 = D^2
\]

Expand and collect like terms:

\begin{align*}
(b_2 x - b_1 y)^2 &= b_2^2 x^2 - 2 b_1 b_2 x y + b_1^2 y^2\\
(a_1 y - a_2 x)^2 &= a_1^2 y^2 - 2 a_1 a_2 x y + a_2^2 x^2
\end{align*}

Adding yields

\[
(b_2^2 + a_2^2) x^2 - 2(b_1 b_2 + a_1 a_2) x y + (b_1^2 + a_1^2) y^2 = D^2
\]

Thus the curve satisfies a quadratic form
\[
A x^2 + B xy + C y^2 = D^2
\]
with
\[
A = a_2^2 + b_2^2\qquad B = -2(a_1 a_2 + b_1 b_2)\qquad C = a_1^2 + b_1^2
\]

To determine the type of conic compute the discriminant $\Delta = B^2 - 4AC$

\begin{align*}
\Delta &= \bigl(-2(a_1 a_2 + b_1 b_2)\bigr)^2 - 4(a_2^2 + b_2^2)(a_1^2 + b_1^2)\\[6pt]
&= 4(a_1 a_2 + b_1 b_2)^2 - 4\bigl(a_1^2 + b_1^2\bigr)\bigl(a_2^2 + b_2^2\bigr)\\[6pt]
&= 4\Bigl((a_1 a_2 + b_1 b_2)^2 - (a_1^2 + b_1^2)(a_2^2 + b_2^2)\Bigr)
\end{align*}

The expression inside the parentheses simplifies to $-(a_1 b_2 - a_2 b_1)^2 = -D^2$ Hence

\[
\Delta = 4(-D^2) = -4D^2 < 0
\]

Since $\Delta<0$ the curve is an ellipse

\newpage
%%%%%%%%%%%%%%%%%%%%%%%%%%%%%%%%%%%%%%%%%%%%%%%%%%%%%%%%%%%%
\item \textbf{Identify the parametrized surface}
\[
\vec{r}(u,v) = \bigl(\sqrt{1+v^2}\cos u,\;\sqrt{1+v^2}\sin u,\; v\bigr)
\]

\textbf{Sol.}

Write the components as $x=\sqrt{1+v^2}\cos u$ $y=\sqrt{1+v^2}\sin u$ $z=v$ Compute $x^2+y^2$

\begin{align*}
x^2 + y^2 &= ( \sqrt{1+v^2}\cos u)^2 + ( \sqrt{1+v^2}\sin u)^2 \\
&= (1+v^2)(\cos^2 u + \sin^2 u) \\
&= 1+v^2
\end{align*}

But $z=v$ so $v^2=z^2$ Substitute:

\[
x^2 + y^2 = 1 + z^2
\]

Rearrange:

\[
x^2 + y^2 - z^2 = 1
\]

This is the equation of a one-sheeted hyperboloid

\newpage
%%%%%%%%%%%%%%%%%%%%%%%%%%%%%%%%%%%%%%%%%%%%%%%%%%%%%%%%%%%%
\item \textbf{Find a bijective continuous map from the cylinder}
\[
\bb S_1 := \{(x,y,z)\in\bb R^3 \mid x^2+y^2 \le 1\}
\]
\textbf{onto}
\[
\bb S_2 := \{(X,Y,Z)\in\bb R^3 \mid X^2+Y^2-Z^2 \le 1\}
\]

\textbf{Sol.}

Define the map $f:\bb S_1\to\bb R^3$ by
\[
f(x,y,z) = \bigl(X,Y,Z\bigr) = \bigl(x\sqrt{1+z^2},\; y\sqrt{1+z^2},\; z\bigr)
\]

\textbf{(i) $f$ maps $\bb S_1$ into $\bb S_2$}

Take $(x,y,z)\in\bb S_1$ so $x^2+y^2\le 1$ Compute

\begin{align*}
X^2 + Y^2 - Z^2
&= (1+z^2)(x^2+y^2) - z^2 \\
&= x^2+y^2 + z^2(x^2+y^2 - 1)
\end{align*}

Because $x^2+y^2-1\le 0$ we have
\[
X^2+Y^2-Z^2 \le x^2+y^2 \le 1
\]

Hence $f(\bb S_1)\subseteq \bb S_2$

\medskip

\textbf{(ii) Injectivity}

Suppose $f(x_1,y_1,z_1)=f(x_2,y_2,z_2)$ Then $z_1=z_2=z$ and
\[
x_1\sqrt{1+z^2} = x_2\sqrt{1+z^2}\qquad
y_1\sqrt{1+z^2} = y_2\sqrt{1+z^2}
\]
Thus $x_1=x_2$ and $y_1=y_2$ so injective

\medskip

\textbf{(iii) Surjectivity}

Let $(X,Y,Z)\in\bb S_2$ Define
\[
(x,y,z) := \Bigl(\frac{X}{\sqrt{1+Z^2}},\; \frac{Y}{\sqrt{1+Z^2}},\; Z\Bigr)
\]
Then
\[
x^2+y^2 = \frac{X^2+Y^2}{1+Z^2} \le 1
\]
and $f(x,y,z)=(X,Y,Z)$ Hence surjective

\medskip

\textbf{(iv) Continuity and inverse}

Both $f$ and
\[
f^{-1}(X,Y,Z) = \Bigl(\frac{X}{\sqrt{1+Z^2}},\; \frac{Y}{\sqrt{1+Z^2}},\; Z\Bigr)
\]
are continuous Hence a homeomorphism

\newpage
%%%%%%%%%%%%%%%%%%%%%%%%%%%%%%%%%%%%%%%%%%%%%%%%%%%%%%%%%%%%
\item \textbf{Vector calculus identities} For a scalar $f:\bb R^3\to\bb R$ and a vector field
$F=(F_1,F_2,F_3):\bb R^3\to\bb R^3$ (assume all components are $C^2$) prove
\[
\nabla\times(\nabla f)=\mathbf{0}\qquad
\nabla\cdot(\nabla\times F)=0
\]

\textbf{Sol.}

\noindent\textbf{(A) $\nabla\times(\nabla f)=\mathbf{0}$}

\[
\nabla f = (f_x,f_y,f_z)
\]

By definition
\[
\nabla\times(\nabla f)
= \begin{vmatrix}
\mathbf{i} & \mathbf{j} & \mathbf{k} \\
\partial_x & \partial_y & \partial_z \\
f_x & f_y & f_z
\end{vmatrix}
= \bigl(f_{z y}-f_{y z},\; f_{x z}-f_{z x},\; f_{y x}-f_{x y}\bigr)
\]

By equality of mixed partial derivatives (Clairaut's theorem) all terms vanish so
\[
\nabla\times(\nabla f) = (0,0,0) = \mathbf{0}
\]

\medskip

\noindent\textbf{(B) $\nabla\cdot(\nabla\times F)=0$}

\[
\nabla\times F =
\begin{vmatrix}
\mathbf{i} & \mathbf{j} & \mathbf{k}\\
\partial_x & \partial_y & \partial_z\\
F_1 & F_2 & F_3
\end{vmatrix}
= (F_{3y}-F_{2z},\; F_{1z}-F_{3x},\; F_{2x}-F_{1y})
\]

Now take divergence:
\[
\nabla\cdot(\nabla\times F) = (F_{3y}-F_{2z})_x + (F_{1z}-F_{3x})_y + (F_{2x}-F_{1y})_z
\]

Expand:
\[
= F_{3yx} - F_{2zx} + F_{1zy} - F_{3xy} + F_{2xz} - F_{1yz}
\]

By equality of mixed partials each pair cancels hence
\[
\nabla\cdot(\nabla\times F)=0
\]

\end{enumerate}
\end{document}
