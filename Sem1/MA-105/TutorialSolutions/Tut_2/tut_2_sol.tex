\documentclass[14pt]{extarticle}
\usepackage[]{graphicx}
\usepackage{amsmath,amssymb,mathtools}
\usepackage{bbm}
\usepackage{amsthm}

\title{MA-105 Tutorial-2 Solutions}
\author{Daksh Maahor}
\date{August 2025}

\begin{document}

\maketitle

\begin{enumerate}

%%%%%%%%%%%%%%%%%%%%%%%%%%%%%%%%%%%%%%%%%%%%%%%%%%%%%%%%%%%%
\item \textbf{Let $x_n = 1 + \frac12 + \dots + \frac1n$. Show that $x_4 > 2$, $x_8 > \frac{5}{2}$ and $x_{16} > 3$}

\begin{proof}
We know the sequence $x_n = \sum_{k=1}^n \frac{1}{k}$ is monotonically increasing

\medskip
\[
x_4 = 1 + \frac12 + \frac13 + \frac14 = 1 + 0.5 + 0.333\ldots + 0.25 \approx 2.083 > 2
\]

\medskip
\[
x_8 = x_4 + \frac15 + \frac16 + \frac17 + \frac18 > 2 + 4 \cdot \frac18 = 2 + 0.5 = 2.5 = \frac{5}{2}
\]

\medskip
\[
x_{16} = x_8 + \sum_{k=9}^{16} \frac1k > \frac{5}{2} + 8 \cdot \frac{1}{16} = \frac{5}{2} + \frac12 = 3
\]

Hence the inequalities hold
\[
x_4>2,\quad x_8 > \frac{5}{2},\quad x_{16} > 3
\]
\end{proof}

\newpage
%%%%%%%%%%%%%%%%%%%%%%%%%%%%%%%%%%%%%%%%%%%%%%%%%%%%%%%%%%%%
\item \textbf{Let $a>0$ and $x_n = 1 + a + \frac{a^2}{2} + \dots + \frac{a^n}{n!}$. Show that $(x_n)$ is monotonically increasing and bounded above}

\begin{proof}

The sequence satisfies 
\[
x_{n+1} = x_n + \frac{a^{n+1}}{(n+1)!} \ge x_n
\]
Hence $(x_n)$ is monotonically increasing

\medskip

Notice that for $a>0$, for $n \geq 2$
\[
x_n = \sum_{k=0}^{n} \frac{a^k}{k!} < 1 + a + \frac{a^2}{2} + \frac{a^3}{4} + \dots = 1 + a + \frac{\frac{a^2}{2}}{1-\frac{a}{2}} = 1+a+\frac{a^2}{2-a}
\]

Hence $(x_n)$ is bounded above by $\frac{2+a}{2-a}$.

\medskip
\textbf{Conclusion:} $(x_n)$ is monotonically increasing and bounded above
\end{proof}

\newpage
%%%%%%%%%%%%%%%%%%%%%%%%%%%%%%%%%%%%%%%%%%%%%%%%%%%%%%%%%%%%
\item \textbf{Is the function $f(x) = \frac{\log(x+1)}{\sin x}$ defined on $\left(-\frac12, \frac12\right)$ differentiable at $0$, given $f(0) = 1$? Find $f'(0)$}

\begin{proof}
Compute the derivative at $0$ using the definition
\[
f'(0) = \lim_{x\to 0} \frac{f(x) - f(0)}{x} = \lim_{x\to 0} \frac{\frac{\log(1+x)}{\sin x} - 1}{x}
= \lim_{x\to 0} \frac{\log(1+x) - \sin x}{x \sin x}
\]

Use the Taylor expansions around $x=0$:
\[
\log(1+x) = x - \frac{x^2}{2} + O(x^3), \qquad \sin x = x - \frac{x^3}{6} + O(x^5)
\]

Thus
\[
\log(1+x) - \sin x = -\frac{x^2}{2} + O(x^3), \quad x\sin x = x^2 + O(x^4)
\]

Hence
\[
f'(0) = \lim_{x\to 0} \frac{-\frac{x^2}{2} + O(x^3)}{x^2 + O(x^4)} = -\frac12
\]

\medskip
\textbf{Conclusion:} $f$ is differentiable at $0$ and
\[
\boxed{f'(0) = -\frac12}
\]
\end{proof}

\newpage
%%%%%%%%%%%%%%%%%%%%%%%%%%%%%%%%%%%%%%%%%%%%%%%%%%%%%%%%%%%%
\item \textbf{Compute $D^n f(1)$ when $f(x) = \frac{(x^2-1)^n}{2^n n!}$}

\begin{proof}
Notice $f(x) = \frac{(x-1)^n (x+1)^n}{2^n n!}$

Apply Leibniz formula:
\[
D^n[(x-1)^n (x+1)^n] = \sum_{k=0}^n \binom{n}{k} D^k (x-1)^n D^{n-k} (x+1)^n
\]

Only $D^n (x-1)^n = n!$, $D^n (x+1)^n = n!$, and $D^k (x-1)^n = 0$ for $k>n$

Evaluate at $x=1$: $(x-1)^n = 0$ so all terms vanish except $k=n$
\[
D^n f(1) = \frac{1}{2^n n!} \cdot \binom{n}{n} n! \cdot D^0 (x+1)^n|_{x=1} = \frac{(1+1)^n}{2^n} = 1
\]

\medskip
\textbf{Conclusion:} $D^n f(1) = 1$
\end{proof}

\newpage
%%%%%%%%%%%%%%%%%%%%%%%%%%%%%%%%%%%%%%%%%%%%%%%%%%%%%%%%%%%%
\item \textbf{Prove that if $f,g$ are two $n$-times differentiable functions at $p$ then $(D^n (fg))(p) = \sum_{k=0}^n \binom{n}{k} D^k f(p) D^{n-k} g(p)$}

\begin{proof}
This is the Leibniz rule. 

\medskip
\textbf{Induction on $n$:}

\textbf{Base case $n=1$:} $(fg)' = f'g + fg'$ holds.

\textbf{Induction step:} Assume formula holds for $n$, then
\begin{align*}
D^{n+1} (fg) &= D(D^n(fg)) \\ &= D\left( \sum_{k=0}^n \binom{n}{k} D^k f \cdot D^{n-k} g \right) \\ &= \sum_{k=0}^n \binom{n}{k} \bigl( D^{k+1} f \cdot D^{n-k} g + D^k f \cdot D^{n-k+1} g \bigr)
\end{align*}

Reindex sums and use $\binom{n}{k} + \binom{n}{k-1} = \binom{n+1}{k}$ to obtain
\[
D^{n+1} (fg) = \sum_{k=0}^{n+1} \binom{n+1}{k} D^k f \cdot D^{n+1-k} g
\]

\medskip
Hence the formula holds for all $n \in \mathbbm{N}$
\end{proof}

\newpage
%%%%%%%%%%%%%%%%%%%%%%%%%%%%%%%%%%%%%%%%%%%%%%%%%%%%%%%%%%%%
\item \textbf{Prove that $D^n \sin x = \sin(x + \frac{n\pi}{2})$}

\begin{proof}
Use induction on $n$:

\medskip
\textbf{Base case $n=0$:} $D^0 \sin x = \sin x = \sin(x + 0)$

\textbf{Induction step:} Assume $D^n \sin x = \sin(x + n\pi/2)$. Then
\begin{align*}
D^{n+1} \sin x &= D(\sin(x+n\pi/2)) \\ &= \cos(x+n\pi/2) \\ &= \sin(x + n\pi/2 + \pi/2) \\ &= \sin(x + (n+1)\pi/2)
\end{align*}

Hence formula holds for all $n \in \mathbbm{N}$
\end{proof}

\newpage
%%%%%%%%%%%%%%%%%%%%%%%%%%%%%%%%%%%%%%%%%%%%%%%%%%%%%%%%%%%%
\item \textbf{Find the $n^{th}$ derivative of $\sin^{-1} x$ at $x=0$}

\begin{proof}
Let $y = \sin^{-1} x$, so $x = \sin y$ and $\frac{dx}{dy} = \cos y$

\medskip
\textbf{Step 1: First derivative:} 
\[
y' = \frac{dy}{dx} = \frac{1}{\cos y} = \frac{1}{\sqrt{1-x^2}}
\]

\textbf{Step 2: Higher derivatives:} By induction, only odd derivatives are nonzero at $x=0$ due to symmetry, and we have
\[
D^{2n} (\sin^{-1} x)(0) = 0
\]

\textbf{Step 3: Odd derivatives:} 
\[
D^{2n+1} (\sin^{-1} x)(0) = \frac{((2n)!)^2}{4^n(n!)^2}
\]

Hence the $n^{th}$ derivative at $0$ is
\[
D^n (\sin^{-1} x)(0) = \begin{cases}
0 & \text{if $n$ even} \\
\dfrac{((2m)!)^2}{4^m (m!)^2} & \text{if $n=2m+1$ odd}
\end{cases}
\]
\end{proof}

\newpage
%%%%%%%%%%%%%%%%%%%%%%%%%%%%%%%%%%%%%%%%%%%%%%%%%%%%%%%%%%%%
\item \textbf{Suppose $y(x)$ is infinitely differentiable and satisfies $(1-x^2)y'' - 2xy' + p(p+1)y = 0$. Prove that if $y(1) = 0$, then $D^n y(1) = 0$ for all $n \in \mathbbm{N}$}

\begin{proof}
\textbf{Step 1: Factor $(1-x^2)$}  

Rewrite the equation:
\[
(1-x)(1+x)y'' - 2xy' + p(p+1)y = 0
\]

\textbf{Step 2: Evaluate at $x=1$}  

At $x=1$, $(1-x^2) = 0$, so the first term vanishes. Using $y(1)=0$, the equation gives
\[
-2 y'(1) + p(p+1) y(1) = -2 y'(1) + 0 = 0 \implies y'(1)=0
\]

\textbf{Step 3: Higher derivatives by induction}  

Assume $D^k y(1) = 0$ for $k \le n$, differentiate the ODE and evaluate at $x=1$. Each differentiation yields a factor $(1-x)$ multiplying the highest derivative, which vanishes at $x=1$, and the lower-order terms vanish by induction hypothesis. Hence $D^{n+1} y(1) = 0$

\medskip
\textbf{Conclusion:} $D^n y(1) = 0$ for all $n \in \mathbbm{N}$
\end{proof}

\end{enumerate}

\end{document}
